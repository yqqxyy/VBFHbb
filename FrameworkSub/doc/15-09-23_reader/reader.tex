\documentclass{beamer}
%\usepackage{beamerthemeshadow}
%\beamersetuncovermixins{\opaqueness<1>{25}}{\opaqueness<2->{15}}
\setbeamertemplate{navigation symbols}{}

\setbeamersize{text margin left=5mm}
\setbeamersize{text margin right=5mm}

\usepackage{color}
\usepackage{amssymb}
\usepackage{rotating}
\usepackage{grffile}
\usepackage{tabularx}

\usepackage{epstopdf}

\pdfpageattr{/Group << /S /Transparency /I true /CS /DeviceRGB>>}

\usepackage{listings}
\lstset{
     language     = C++,
     basicstyle   = \ttfamily,
     keywordstyle = \ttfamily\color[rgb]{0.0, 0.0, 1.0},
     stringstyle  = \ttfamily\color[rgb]{1.0, 0.0, 0.0},
     commentstyle = \ttfamily\color[rgb]{0.0, 0.6, 0.0},
     backgroundcolor = \color[rgb]{0.9, 0.9, 0.9},
     frame        = single,
     numbers      = left,
}

\usetheme{Madrid}
\usecolortheme{seahorse}

\definecolor{titlecolorr}{rgb}{0.75,0.8,0.9}
\definecolor{blockcolorr}{rgb}{0.3,0.35,0.55}
\setbeamercolor*{frametitle}{bg=titlecolorr,fg=black}
\setbeamercolor{author in head/foot}{bg=titlecolorr,fg=black}
\setbeamercolor{title in head/foot}{bg=titlecolorr,fg=black}
\setbeamercolor{title}{bg=blockcolorr,fg=white}
\setbeamercolor{block title}{bg=blockcolorr,fg=white}
\setbeamercolor{date in head/foot}{bg=titlecolorr,fg=black}
%\setbeamerfont{slide title}{size=\LARGE,series=\bfseries}

\begin{document}

% \renewcommand{\inserttotalframenumber}{2}

\newcommand{\newImage}[2]{
  \begin{minipage}{#1\textwidth}
  \renewcommand{\_}{_}
  \IfFileExists{#2}
  {\includegraphics[width=1\textwidth]{#2}}
  {\centering N/A}
  \renewcommand{\_}{\textunderscore}
    \end{minipage}
}

\title{CxAOD structure and CxAODReader}
\author[Daniel B\"uscher]{
%Hannah Arnold, \underline{
Daniel B\"uscher%}, Christian Weiser
}
\institute{Universit\"at Freiburg}\date{\today} 

\frame{\titlepage}

%\footnotesize
\scriptsize

\begin{frame}[fragile]
\frametitle{Introduction}
% \structure{Inputs}
\begin{itemize}
 \item Discussing the object retrieval and systematic loop in CxAODReader
 \item Showing example code from SVN (links below), but simplified in many places
 \item[$\Rightarrow$] The SVN code has more checks and optimizations, which should be kept
\end{itemize}
\begin{center}
\end{center}
{\tiny
Code from:
\vspace{-2mm}
\begin{verbatim}
https://svnweb.cern.ch/trac/
atlasoff/browser/PhysicsAnalysis/HiggsPhys/Run2/Hbb/CxAODFramework/CxAODReader/
https://svnweb.cern.ch/trac/
atlasphys-exa/browser/Physics/Exotic/Analysis/DibosonResonance/Data2015/Code/CxAODFramework_DB/CxAODReader_DB/
\end{verbatim}
Basic info on the xAOD EDM:
\vspace{-2mm}
\begin{verbatim}
https://twiki.cern.ch/twiki/bin/view/AtlasComputing/SoftwareTutorialxAODAnalysisInROOT
\end{verbatim}
Tutorial slides with more info on CxAOD structure:
\vspace{-2mm}
\begin{verbatim}
https://svnweb.cern.ch/trac/atlasoff/export/HEAD/PhysicsAnalysis/HiggsPhys/Run2/Hbb/CxAODFramework/
FrameworkSub/trunk/doc/15-02-17_tutorial/CxAODtutorial.pdf
\end{verbatim}
}
\end{frame}

\begin{frame}[fragile]
\frametitle{CxAOD structure}
% \structure{Inputs}
\begin{itemize}
 \item Systematic variations of objects (particles) are stored as shallow copies (SCs) of Nominal
\begin{itemize}
\scriptsize
 \item Only the Nominal collection contains all branches
 \item SCs contain only those branches affected by systematics
 \item[$\Rightarrow$] Advantage in disk size and CPU time over conventional Ntuples
\end{itemize}
\item Example for jets:
\begin{itemize}
\scriptsize
 \item \textcolor{blue}{Four-vectors and selection flags} are stored for Nominal as well as for the variation
 \item \textcolor{red}{``jvf0'' and ``nrElectron/MuonInJet''} are only stored for Nominal jets
 \item Note: CxAODs provide an easy drawing of systematic effects!
\end{itemize}
\end{itemize}
\begin{center}
\newImage{0.95}{browser.png}
\end{center}
\end{frame}

\begin{frame}[fragile]
\frametitle{xAOD EDM and implications}
% \structure{Inputs}
\begin{itemize}
 \item At first look the structure of the CxAOD seems rather complicated
 \item However,the xAOD EDM provides convenient way to read variations:
\begin{itemize}
\scriptsize
 \item SCs of objects can be used exactly as the Nominal ones
 \item[$\Rightarrow$] Both can be accessed trough e.g. xAOD::Jet
 \item If one tries to read a non-exising branch of the SC the Nominal value is returned
%  \item[$\Rightarrow$] E.g. for jvf0
\end{itemize}
\item This implies that additional objects have to be stored
\begin{itemize}
\scriptsize
 \item If an object is stored for some variation it has to be stored for all variations \\
 (e.g. if a jet passed selection for JES-up, it is stored for Nominal, JES-do, ...) 
 \item[$\Rightarrow$] The CxAOD can contain objects that did not actually pass the pre-selection
 \item The same applies to events
\end{itemize}
\item[$\Rightarrow$] Object and event pre-selection have to be re-applied when reading the CxAOD!
\begin{itemize}
\scriptsize
\item Objects can be re-selected using flags + re-applying overlap removal
\item For event selection the same routine can be re-applied
\item[$\Rightarrow$] Both implemented by CxAODReader
\end{itemize}
\end{itemize}
\end{frame}

\begin{frame}[fragile]
\frametitle{Initialize selection in CxAODReader}
% \structure{Inputs}
\begin{itemize}
 \item AnalysisReader::initizializeSelection() determines which event selection class is used
 \item[$\Rightarrow$] Should be the same as for producing the CxAOD
 \item In addition, a ``fill function'' is defined: fills histograms, output trees, ...
\end{itemize}
\begin{center}
\begin{minipage}{0.9\textwidth}
{\tiny
\begin{lstlisting}
EL::StatusCode AnalysisReader_DB :: initializeSelection() {
  m_config->getif<int>("analysisType", m_analysisType);
  if (m_analysisType==VH1LepHist) {
    m_eventSelection = new lvqq_MainEvtSelection(m_config); 
    m_fillFunction = std::bind(&AnalysisReader_DB::fill_VH1LepHist, this);
  }
  // initialize other selections, MVATree, ...
}

\end{lstlisting}
}
\end{minipage}
\end{center}
\end{frame}

\begin{frame}[fragile]
\frametitle{The event loop in CxAODReader}
% \structure{Inputs}
\begin{itemize}
 \item For each event a loop on the variations is performed (different to most Run 1 codes)
\begin{itemize}
\scriptsize
 \item[{\bf (a)}] ObjectReader take care of reading the xAOD containers (via TEvent::retrieve())
 \item[{\bf (b)}] The overlap removal and event selection are applied
 \item[{\bf (c)}] A user defined fill function is called
\end{itemize}
 \item Note: (b), (c) don't need to know which variation is processed, \\
 they are just provided with objects of a given variation
\end{itemize}
\begin{center}
\begin{minipage}{0.9\textwidth}
{\tiny
\begin{lstlisting}
EL::StatusCode AnalysisReader::execute () {
  for (std::string varName : m_variations) {
    m_electrons = m_electronReader->getObjects(varName);
    m_jets      = m_jetReader->getObjects(varName);
    // ... retrieve other containers ...
    setObjectsForOR(m_electrons, ...);
    m_overlapRemoval->removeOverlap(m_electrons, ...)
    bool passSel = m_eventSelection->passPreSelection(m_electrons, ...);
    passSel     &= m_eventSelection->passSelection(m_electrons, ...);
    if (!passSel) continue;
    m_fillFunction();
  }
}
\end{lstlisting}
}
\end{minipage}
\end{center}
\end{frame}

\begin{frame}[fragile]
\frametitle{The fill function}
% \structure{Inputs}
\begin{itemize}
 \item The fill function retrieves the result of the event selection
 \item[$\Rightarrow$] The result contains the selected objects for the present variation
 \item One could directly access m\_jets, but these are not guaranteed have passed selection!
 \item The xAOD objects provide methods retrieving four-vectors and some other properties
 \item Other values can be read using PROPs (defined in CommonProperties.h or anywhere else)
 \item[$\Rightarrow$] Analysis variables can be calculated and filled into histograms, ...
\end{itemize}
\begin{center}
\begin{minipage}{0.9\textwidth}
{\tiny
\begin{lstlisting}
EL::StatusCode AnalysisReader_DB::fill_VH1LepHist() {
  DBResult selectionResult = ((lvqq_MainEvtSelection*)m_eventSelection)->result();
  std::vector<const xAOD::Electron*> electrons = selectionResult.el;
  std::vector<const xAOD::Jet*>      largeJets = selectionResult.largeJets;
  std::vector<const xAOD::Jet*>      trackJets = selectionResult.trackJets;
  // ... retrieve other containers ...
  TLorentzVector lepVec  = electron.at(0)->p4();
  TLorentzVector higgVec = largeJets.at(0)->p4();
  float btagVal          = Props::MV2c20.get(trackJets.at(0));
  // ... compute analysis variables, fill histograms, output tree ...
}
\end{lstlisting}
}
\end{minipage}
\end{center}
\end{frame}

\begin{frame}[fragile]
\frametitle{Summary}
\structure{Summary}
\begin{itemize}
 \item The CxAOD format provides an efficient way to store and read systematic variations
 \item Not all stored objects/events are guaranteed to have passed pre-selection
 \item[$\Rightarrow$] Selection has to be re-applied!
 \item This is implemented by CxAODReader 
\begin{itemize}
\scriptsize
 \item Provides an efficient systematics loop
 \item Final analysis variables can be easily calculated using xAOD objects
\end{itemize}
\end{itemize}
\end{frame}

% 
% \begin{frame}
% \frametitle{BACKUP}
% \begin{center}
%  \huge BACKUP
% \end{center}
% \end{frame}


\end{document}


