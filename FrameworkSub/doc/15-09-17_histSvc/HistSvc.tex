\documentclass{beamer}
%\usepackage{beamerthemeshadow}
%\beamersetuncovermixins{\opaqueness<1>{25}}{\opaqueness<2->{15}}
\setbeamertemplate{navigation symbols}{}

\setbeamersize{text margin left=5mm}
\setbeamersize{text margin right=5mm}

\usepackage{color}
\usepackage{amssymb}
\usepackage{rotating}
\usepackage{grffile}
\usepackage{tabularx}

\usepackage{epstopdf}

\usepackage{listings}
\lstset{
     language     = C++,
     basicstyle   = \ttfamily,
     keywordstyle = \ttfamily\color[rgb]{0.0, 0.0, 1.0},
     stringstyle  = \ttfamily\color[rgb]{1.0, 0.0, 0.0},
     commentstyle = \ttfamily\color[rgb]{0.0, 0.6, 0.0},
     backgroundcolor = \color[rgb]{0.9, 0.9, 0.9},
     frame        = single,
     numbers      = left,
}

\usetheme{Madrid}
\usecolortheme{seahorse}

\definecolor{titlecolorr}{rgb}{0.75,0.8,0.9}
\definecolor{blockcolorr}{rgb}{0.3,0.35,0.55}
\setbeamercolor*{frametitle}{bg=titlecolorr,fg=black}
\setbeamercolor{author in head/foot}{bg=titlecolorr,fg=black}
\setbeamercolor{title in head/foot}{bg=titlecolorr,fg=black}
\setbeamercolor{title}{bg=blockcolorr,fg=white}
\setbeamercolor{block title}{bg=blockcolorr,fg=white}
\setbeamercolor{date in head/foot}{bg=titlecolorr,fg=black}
%\setbeamerfont{slide title}{size=\LARGE,series=\bfseries}

\begin{document}

% \renewcommand{\inserttotalframenumber}{2}

\newcommand{\newImage}[2]{
  \begin{minipage}{#1\textwidth}
  \renewcommand{\_}{_}
  \IfFileExists{#2}
  {\includegraphics[width=1\textwidth]{#2}}
  {\centering N/A}
  \renewcommand{\_}{\textunderscore}
    \end{minipage}
}

\title{Histogram handling in CxAODReader}
\author[Daniel B\"uscher]{
%Hannah Arnold, \underline{
Daniel B\"uscher%}, Christian Weiser
}
\institute{Universit\"at Freiburg}\date{\today} 

\frame{\titlepage}

%\footnotesize
\scriptsize

\begin{frame}[fragile,t]
\frametitle{Introduction}
% \structure{Inputs}
\begin{itemize}
 \item Discussing histogram handling in CxAODReader using the example of the VHreso code
 \item Naming convention: ``channel\_tags\_description/variation/sample\_distribution''
 \item Showing example code from SVN (links below), but simplified in many places
 \item[$\Rightarrow$] The SVN code has more checks and optimizations, which should be kept
\end{itemize}
\begin{center}
\newImage{0.8}{browser1.png}
\end{center}
{\tiny
\begin{verbatim}
https://svnweb.cern.ch/trac/
atlasoff/browser/PhysicsAnalysis/HiggsPhys/Run2/Hbb/CxAODFramework/CxAODReader/
https://svnweb.cern.ch/trac/
atlasphys-exa/browser/Physics/Exotic/Analysis/DibosonResonance/Data2015/Code/CxAODFramework_DB/CxAODReader_DB/
\end{verbatim}
}
\end{frame}

\begin{frame}[fragile,t]
\frametitle{HistSvc and HistNameSvc initialization}
% \structure{Inputs}
\begin{itemize}
 \item HistSvc: generic histogram handling
 \item HistNameSvc: analysis specific histogram naming
 \item[$\Rightarrow$] The VHreso code uses a custom HistNameSvc\_VHreso
 \item Both are initialized in AnalysisReader\_DB::histInitialize()
 \item The HistSvc object gets a pointer to the HistNameSvc object
\end{itemize}
\begin{center}
\begin{minipage}{0.9\textwidth}
{\tiny
\begin{lstlisting}
EL::StatusCode AnalysisReader_DB :: histInitialize () {
  // ... do some stuff ...
  m_histSvc = new HistSvc();
  if (m_analysisType == VH1LepHist) {
    m_histNameSvc = new HistNameSvc_VHreso();
  } else {
    m_histNameSvc = new HistNameSvc();
  }
  m_histSvc -> SetNameSvc(m_histNameSvc);
}
\end{lstlisting}
}
\end{minipage}
\end{center}
\end{frame}


\begin{frame}[fragile,t]
\frametitle{Histogram booking and filling}
% \structure{Inputs}
\begin{itemize}
 \item HistSvc provides 1-line booking and filling of histograms (lines 11-12)
 \item Only the distibution name is provided when filling
 \item The full histogram name is determined by HistNameSvc (info given beforehand)
 \item[$\Rightarrow$] E.g. number of tags (nTag, nAddTag) are set before filling (lines 8-9)
 \item Note that such information can be provided to HistNameSvc anywhere in the reader
 \item[$\Rightarrow$] E.g. in a separate b-tagging method
\end{itemize}
\begin{center}
\begin{minipage}{0.9\textwidth}
{\tiny
\begin{lstlisting}
EL::StatusCode AnalysisReader_DB :: fill_VH1LepHist() {
  // ... get containers from selection result ...
  int nTag = 0;
  for(const xAOD::Jet* trackJet : matchedJets){
    if (Props::MV2c20.get(trackJet) > bTagCut) nTag++;
  }
  // ... calculate nAddTag ...
  ((HistNameSvc_VHreso*) m_histNameSvc) -> set_nTag(nTag);
  ((HistNameSvc_VHreso*) m_histNameSvc) -> set_nAddTag(nAddTag);
  // ... calculate four vectors ...
  m_histSvc -> BookFillHist("vh_m", 200, 0, 4000, VHVec.M()/1e3, m_weight);
  m_histSvc -> BookFillHist("h_pt", 200, 0, 4000, HVec.Pt()/1e3, m_weight);
}
\end{lstlisting}
}
\end{minipage}
\end{center}
\end{frame}


\begin{frame}[fragile,t]
\frametitle{Inner workings of HistSvc}
% \structure{Inputs}
\begin{itemize}
 \item BookFillHist is basically one line: call BookHist and fill the returned histogram
 \item BookHist retrieves the full histogram name from HistNameSvc (line 7)
 \item Then it tries to find the requested histogram in a map and return it
 \item If the histogram is not found (the first time it is requested) a new one is created
\end{itemize}
\begin{center}
\begin{minipage}{0.9\textwidth}
{\tiny
\begin{lstlisting}
void HistSvc::BookFillHist(const string& name, /* ... */) {
  BookHist(name, nbinsx, xlow, xup) -> Fill(value, weight);
  // ... additionally fill weight systs ...
}

TH1* HistSvc::BookHist(const string& name, /* ... */) {
  string fullname = m_nameSvc -> getFullHistName(name);
  TH1* hist = FindHistInMap(fullname);
  if(!hist) {
    hist = new TH1F(fullname, fullname, nbinsx, xlow, xup);
    m_hists[fullname] = hist;
  }
  return hist;
}
\end{lstlisting}
}
\end{minipage}
\end{center}
\end{frame}


\begin{frame}[fragile,t]
\frametitle{Implementation of custom HistNameSvc}
% \structure{Inputs}
\begin{itemize}
 \item The implemention of HistNameSvc\_VHreso determines the full histogram name
 \item[$\Rightarrow$] Slashes are interpreted by EventLoop/SampleHandler to create folder structure
 \item The name is built in getFullHistName() using given information
 \item[$\Rightarrow$] E.g. ``\_2tag'' is added if HistNameSvc::set\_nTag(2) was called beforehand (line 6)
 \item Note that getFullHistName() is called many times $\Rightarrow$ needs efficient implementation
\end{itemize}
\begin{center}
\begin{minipage}{0.9\textwidth}
{\tiny
\begin{lstlisting}
std::string HistNameSvc_VHreso::getFullHistName(const std::string& variable) {
  m_name = "lvJ";
  if (m_nTag < 0) m_name += "_0ptag";
  else if (m_nTag == 0) m_name += "_0tag";
  else if (m_nTag == 1) m_name += "_1tag";
  else if (m_nTag == 2) m_name += "_2tag";
  else m_name += "_3ptag";
  // set nAddTag and description string ...
  m_name += "/" + m_variation + "/";
  m_name += m_sample + "_" + variable;
  return m_name;
}
\end{lstlisting}
}
\end{minipage}\\
\newImage{0.35}{browser2.png}
\end{center}
\end{frame}


\begin{frame}[fragile]
\frametitle{Summary}
\structure{Summary}
\begin{itemize}
 \item HistSvc provides a convenient way for histogram booking and filling (1-liner)
 \item HistNameSvc separates the histogram naming from the reader, can easily be costumized
\end{itemize}
\structure{Not discussed}
\begin{itemize}
 \item Efficient filling of weight systematics with HistSvc, some info at:
{\tiny
\begin{verbatim}
https://its.cern.ch/jira/browse/CXAOD-24
\end{verbatim}
}
 \item Filling the same distribution multiple times in one event (e.g. for different regions)
 \item[$\Rightarrow$] Can mostly be avoided by a good definition of orthogonal regions
 \item[$\Rightarrow$] If needed on can do a simple loop over BookFillHist, example at:
{\tiny
\begin{verbatim}
https://svnweb.cern.ch/trac/atlasphys-exa/browser/Physics/Exotic/Analysis/DibosonResonance/Data2015
/Code/CxAODFramework_DB/CxAODReader_DB/trunk/Root/AnalysisReader_DB.cxx?rev=240346#L927
\end{verbatim}
}
\end{itemize}
\end{frame}

% 
% \begin{frame}
% \frametitle{BACKUP}
% \begin{center}
%  \huge BACKUP
% \end{center}
% \end{frame}


\end{document}


